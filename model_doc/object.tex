Les instances prises en compte sont le prévisionnel
(\forecast), le compte en banque (\bank) et le 
liquide disponible (\liquid). Une opération est un flux d'argent, 
entrant ou sortant du compte ou du liquide. Les combinaisons possibles 
sont entre le compte et le liquide, ou à l'intérieur du compte entre le 
compte courant et un ou plusieurs compte-épargne (\saving).

Une prévision se fait sur un pas de temps prédéfinis, les sommes
étant prévues pour une période fixe. Cette période est par convention
dans ce programme le mois.

\begin{figure}
\centering
\frame{\includegraphics{entite}}
\caption{\label{compta:entites}Les différents objects et leur interactions.}
\end{figure}

\subsection{Prévisionnel}

Prévoir les dépenses peut se faire en les classant et assignant
un plafond à ne pas dépasser. La prédiction étant un art difficile,
et surtout lorsqu'il s'agit du futur\footnote{\og Prediction is very
difficult, especially about the futur.\fg\ \textsc{Niels Bohr}
(1885--1962)}, ce plafond peut se nuancer d'une certaine marge
pour le rendre flottant.
Ainsi chaque catégorie se caractérise par un nom, un plafond, mais
aussi une marge de man\oe uvre.

Il est possible (et même fortement probable) qu'il existe des
opérations récurrentes tous les mois, typiquement des factures, 
un loyer, un prêt, etc. Ainsi il faut pouvoir ajouter
à certaines catégories des opérations qui, par hypothèse,
se passeront à un moment ou un autre dans le mois.
Donc il est possible d'ajouter des opérations dans chaque
catégorie.

Le prévisionnel se décompose en catégories. Le but étant de classer 
l'ensemble des opérations par catégorie, de définir un plafond par 
catégorie et de comparer les dépenses et recettes réelles aux 
provisions allouées à la catégorie. Une catégorie peut comprendre 
des opérations attendues, comme par exemple un prêt ou un loyer à 
payer, ou n'importe quel prélèvement automatique.

Une catégorie peut donc contenir donc des opérations, caractérisées
par:
\begin{itemize}
\item un montant;
\item une marge;
\item une date de début effectif;
\item une date de fin effective;
\item une variable logique déterminant si cette sous-catégorie
        caractérise une opération automatique;
\item dans le cas d'une opération automatique, une période, en mois,
        déterminant à quels mois cette opération est attendue.
\end{itemize}

%\begin{figure}[htp]
%\centering
%\includegraphics[width=\linewidth]{forecast}
%\caption{\label{forecast}Forecast}
%\end{figure}

\subsection{Compte bancaire}
\subsection{Compte épargne}
\subsection{Liquide}

