Une comptabilité se fait entre des comptes bancaires, des comptes d'épargne
associés et des liquidités. Les comptes d'épargne dépendent entièrement d'un
compte courant (appelé bancaire ici) auquel ils sont associés. Les flux d'argent
se passent entre l'extérieur et un compte (utilisation de carte bleue), un compte
et une liquidité (retrait d'un compte), un compte et une épargne associée
(voir Fig.~\ref{compta:comptes}).
\begin{figure}
\centering
\includegraphics{entite}
\caption{\label{compta:comptes}Les comptes sont répartis entre comptes bancaires et
        comptes liquides. Tant qu'ils sont dans la même monnaie, ils peuvent tous
        interagir les uns avec les autres. Les comptes épargnes sont totalement gérés
        par le compte courant associé, rien ne peut en sortir/entrer sans passer
        par le compta bancaire.}
\end{figure}

Une tenue de comptabilité va consister à comptabiliser ces flux, les organiser
en catégories et comparer le total des flux avec des seuils. Il faut donc
un objet pour définir ces catégories et seuil: le prévisionnel.

\subsection{Prévisionnel}

Prévoir les dépenses peut se faire en les classant et assignant
un seuil à ne pas dépasser. La prédiction étant un art difficile,
et surtout lorsqu'il s'agit du futur\footnote{\og Prediction is very
difficult, especially about the futur.\fg\ \textsc{Niels Bohr}
(1885--1962)}, ce seuil peut se nuancer d'une certaine marge
pour le rendre flottant.
Ainsi chaque catégorie se caractérise par un nom, un seuil, mais
aussi une marge de man\oe uvre.

Il est possible (et même fortement probable) qu'il existe des
opérations récurrentes tous les mois, typiquement des factures, 
un loyer, un prêt, etc. Ainsi il faut pouvoir ajouter
à certaines catégories des opérations qui, par hypothèse,
se passeront à un moment ou un autre dans le mois.
Donc il est possible d'ajouter des opérations dans chaque
catégorie.

Un cran plus loin, il existe des opérations automatiques qui n'ont lieu qu'une
fois tous les deux mois, ou tout les $n$ mois de façon générale.
Il faut donc à chacune de ses opérations associer une période en
mois à laquelle il faut s'attendre à cette opération. Afin de bien
caractériser ce genre de comportement, il faut donc:
\begin{itemize}
\item une date de départ;
\item une date de fin;
\item une période.
\end{itemize}

Au final, le prévisionnel est donc un ensemble de catégories, qui
peuvent contenir des sous-catégories afin de mieux classer. Par
exemple on peut définir une catégorie \catForecast{Administratif}
qui regrouperait les sous-catégories \catForecast{Transport},
\catForecast{Nourriture}, etc. Il est aussi tout à fait possible
de ne vouloir qu'un type de transaction dans une catégorie, et
donc n'avoir aucune sous-catégorie.

\`A cela, nous pouvons ajouter des opérations automatiques, par
exemple payer le loyer. Le loyer est payé tous les mois, il faut
donc s'y attendre s'il n'est pas passé. Un loyer ne dure que la période
pendant laquelle nous sommes locataires, autremement dit, une date de
début, une date de fin. De la même façon, la facture d'électricité doit
être payée, mais des fois tous les deux mois. Au final, notre catégorie
\catForecast{Administratif} peut se décrire ainsi:
\begin{itemize}
\item \catForecast{Transport};
\item \catForecast{Nourriture};
\item \catForecast{Loyer}, de montant fixe, prévu tous les mois, pendant la période
                           de location;
\item \catForecast{Facture de gaz}, de montant prévu fixe, tous les deux mois,
                                    pendant la période dans un logement particulier.
\end{itemize}



Le prévisionnel se décompose en catégories. Le but étant de classer 
l'ensemble des opérations par catégorie, de définir un seuil par 
catégorie et de comparer les dépenses et recettes réelles aux 
provisions allouées à la catégorie. Une catégorie peut comprendre 
des opérations attendues, comme par exemple un prêt ou un loyer à 
payer, ou n'importe quel prélèvement automatique.

Une catégorie peut donc contenir donc des opérations, caractérisées
par:
\begin{itemize}
\item un montant;
\item une marge;
\item une date de début effectif;
\item une date de fin effective;
\item une variable logique déterminant si cette sous-catégorie
        caractérise une opération automatique;
\item dans le cas d'une opération automatique, une période, en mois,
        déterminant à quels mois cette opération est attendue.
\end{itemize}

\subsection{Compte bancaire}
\subsection{Compte épargne}
\subsection{Liquide}

